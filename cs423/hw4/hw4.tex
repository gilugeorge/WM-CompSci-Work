\documentclass[11pt]{article}

\usepackage{exscale}
\usepackage{graphicx}
\usepackage{amsmath}
\usepackage{latexsym}
\usepackage{times,mathptm}
\usepackage{epsfig}
\usepackage{setspace}

\textwidth 6.5truein          
\textheight 9.0truein
\oddsidemargin 0.0in
\topmargin -0.6in

\parindent 0pt          
\parskip 5pt
\def\baselinestretch{1.1}

\begin{document}

\begin{LARGE}
\centerline {\bf CSci 423 Homework 4}
\end{LARGE}
\vskip 0.25cm

\centerline{Due: 1:00 pm, Wednesday, 10/10}
\centerline{Eric Shih}

\begin{enumerate}
 \item (8 points) Prove that the language A of strings of 0s and 1s whose length is a perfect square is not regular
  \begin{description}
    \item We will use proof by contradiction. Assume that A is a regular language, therefore the pumping lemma applies.
    \item Let $p$ be the constant in the pumping lemma. Select $s = (0 \cup 1)^{p^2}$. $|s| \geq p, s = xyz, |xyz|=p^2,
	  |y| > 0$, and $|xy| \leq p$.
    \item According to the pumping lemma, $\forall i \leq 0$, $xy^iz \in A$. Therefore when y is pumped, $s$ should
	  still be a perfect square, thus $|y|$ to any perfect square should result in the next perfect square.
    \item $|s| = p^2$, so the next perfect square will be $(p+1)^2$; letting $i = 2$: 
	  $p^2 + 1 \leq |xy^2z| \leq p^2 + p$ since $1 \leq |y| \leq p$.
    \item $|xy^2z|$ can't be the correct perfect square $p^2$ nor can it be the next perfect square $(p+1)^2$ ($p^2 + 2p +1$)
	  since $|xy^2z| \leq p^2 + p \leq p^2 + 2p +1$. This is a contradiction and the pumping lemma does not apply. Thus
	  we have shown that A is not regular.
  \end{description}

 \item (8 points) Problem 1.46 (c) on page 90.
  \begin{description}
   \item Consider $\overline{L}$ where all strings are palindromes. By the closure properties, if $\overline{L}$ is
	 not regular, $L$ is not regular.
   \item Assume $\overline{L}$ is regular, so the pumping lemma applies. Let $p$ be the constant in the pumping lemma
	 and $s=xyz$ where $\forall i \leq 0$, $xy^iz \in \overline{L}, |y| > 0$, and $|xy| \leq p$.
   \item Let $s = 0^p10^p$, $y$ will be a substring in the first half of $s$ containing at least one 0 and up to $p$
	 number of $0$s. $z$ will be $0*10^p$. When $i=0$, there will be at least one 0 removed from the first part of
	 $x$, leaving $xz$ resulting in anywhere from $0^{p-1}10^p$ to $10^p$, none of which are palindromes.
   \item Thus the pumping lemma does not hold, proving that $\overline{L}$ is not regular, and by closure $L$.
  \end{description}

 \item (16 points) Problem 1.49 (a) and (b) on page 90.\\ a) Strings accepted in B are:
  \begin{center}
    \begin{tabular}{l | l}
      k & $1^ky$ \\ \hline
      1 & $1 \dots 1 \dots$ \\
      2 & $11 \dots 1 \dots 1 \dots$ \\
      3 & $111 \dots 1 \dots 1 \dots 1 \dots$ \\
      4 & $1111 \dots 1 \dots 1 \dots 1 \dots 1 \dots$ \\
    \end{tabular}
  \end{center}
    $\forall w \in B$, $w=1^ky=11^{k-1}y$. From this $w$ can give the regular expression 
    $1(0 \cup 1)^*1(0 \cup 1)^*$ proving $B$ to be regular. \\
    
  b) Show the C is not regular: \\
    \begin{description}
     \item Assume C is regular, therefore the pumping lemma applies. Let $p$ be the pumping length.
     \item We choose $s = 1^p01^p$, $|s| = 2p + 1 \geq p$. Since $|xy| \leq p$, $|xy|$ will
     represent $1^p$ and $z$ will represent $01^p$. When $i = 0$, $y$ is removed from $xyz$.
     This resulting string $xz$ will have $k$ decremented by at least one since $|y| > 0$, but $y$
     will contain the same amount of $1s$ since $z$ is unchanged. Thus $y$ will have more than $k$ $1s$
     resulting in the string not being within C. Thusly, the PL does not hold showing $C$ to not be
     regular.
    \end{description}

 \item (8 points) Problem 1.54 (a) on page 91.
   \begin{description}
    \item F can be defined by 3 subsets:
      \begin{itemize}
      \item $i = 0$, $F_0 = b^*c^*$, this subset is regular.
      \item $i = 1$, $F_1 = ab^nc^n$, found to be not regular in a previous proof.
      \item $i \geq 2$, $F_2 = aa^+b^*c^*$, this subset is regular.
      \end{itemize}
    \item Assuming F is regular, $F = F_0 \cup F_1 \cup F_2$ and each subset is mutually disjoint. This leads to
	  the conclusion that $F_1 = F - F_0 - F_2$. By the closure property, if $F$, $F_0$, and $F_2$ are regular
	  then $F_1$ should be as well. But this is not the case; we know that $F_1$ is not regular from a previous
	  proof, thus providing a contradiction, which in turn proves that F is not regular.
   \end{description}
\end{enumerate}

\pagebreak
\setlength{\parindent}{1cm}
\centerline{\bf Reading Summary 2}

\begin{spacing}{1.5}
 Regular expressions are used for applications that search for patterns in text. Compiled into deterministic or 
 non-deterministic automata, they are used to produce a program that recognizes patterns in text. Two of the most
 important classes of regular expression-based applications are lexical analyzers and text searches. UNIX notation
 is commonly used  to make the applications for regular expressions easier. The UNIX extensions of character 
 classes allow large sets of characters to be represented very succinctly.  \\
 \indent The lexical analyzer scans the source program and recognizes all tokens, such as keywords and identifiers. A 
 lexical-analyzer generator takes a high-level description of a lexical analyzer and produces a functioning lexical
 analyzer. lex and flex are examples of these sort of operations. They are very useful because they are able to
 generate an efficient function that breaks source programs into tokens. This shortens the amount of time that is
 spent generating a proper lexical analyzer. The conversion of the expressions to an automaton begins by building
 an automaton for all expressions. This only tells us that a token has been recognized, but the use of the NFA
 epsilon state, exactly which specific token can be recognized. One simple problem to this strategy is that more
 than one token can be recognized, but this can be solved by giving priority to the first expression listed. \\
 \indent Searching for patterns in text is another popular use for regular-expression notation. Despite not being
 as developed as lexical analyzers, using regular expressions for text searches will offer efficient automaton-based
 implementations. This technology is normally used to describe classes of patterns at a high level, and to allow
 for easy modifications. In the case of a Web application, usually a large amount of text needs to be scanned in
 order to figure out what a query is asking. In the example of finding a street address, a regular expression can
 be used to find what a street is. A set of regular expressions can be used to define what a street is, but can
 also easily be modified to add terms to the recognizer to improve its effectiveness.
 \indent Regular expressions are a powerful tool in producing automated programs. Two very commonly used applications
 are lexical analyzers and text searches. Regular expressions are used in lexical analyzers to simplify the 
 generation of tokens. Text searches use regular expressions to search for wanted patterns in large amounts of text.
 Both applications use regular expressions to simplify the creation of the individual program, but also allows for
 easy modifications or changes.
\end{spacing}



\end{document}
