\documentclass[11pt]{article}

\usepackage{exscale}
\usepackage{graphicx}
\usepackage{amsmath}
\usepackage{latexsym}
\usepackage{times,mathptm}
\usepackage{epsfig}
\usepackage{setspace}

\textwidth 6.5truein          
\textheight 9.0truein
\oddsidemargin 0.0in
\topmargin -0.6in

\parindent 0pt          
\parskip 5pt
\def\baselinestretch{1.1}

\begin{document}

\begin{LARGE}
\centerline {\bf CSci 423 Homework 8}
\end{LARGE}
\vskip 0.25cm

\centerline{Due: 1:00 pm, Wednesday, 11/7}
\centerline{Eric Shih}

\begin{enumerate}
  \item (10 points) Exercise 3.1 (d) on page 159.
    \begin{center}
     Turing Machin $M$ = On input string w:
     \begin{itemize}
      \item{1.} Sweep left to right across the tape, crossing off every other 0.
      \item{2.} If in state 1, the tape contained a single 0, accept.
      \item{3.} If in stage 1, the tape contained more than a single 0 and the number of 0s was odd, reject.
      \item{4.} Return to state 1.
     \end{itemize}
    \end{center}
    
     The configuration for TM $M$ for input string 000000 is: \\
     $\sqcup$q1000000 \\ $\sqcup$q20000 \\ $\sqcup$Xq30000 \\ $\sqcup$X0q4000 \\ $\sqcup$X0Xq300 \\ $\sqcup$X0X0q40 \\ $\sqcup$X0X0Xq3 \\
     $\sqcup$X0X0q50 \\ $\sqcup$X0Xq5X0 \\ $\sqcup$X0q50X0 \\ $\sqcup$Xq5X0X0 \\ $\sqcup$q50X0X0 \\ $\sqcup$q5X0X0X0 \\ $\sqcup$q20X0X0 \\
     $\sqcup$Xq3X0X0 \\ $\sqcup$XXq30X0 \\ $\sqcup$XXXq3X0 \\ $\sqcup$XXX0q40 \\ $\sqcup$XXX0X $q_{reject}$

  \item (10 points) Exercise 3.8 (b) on page 160.
    \begin{enumerate}
      \item Search from left-to-right for a 0; when one is found, write x on the cell
      \item Go back to the beginning and search for a 1, if one is found, write x on the cell; if none is found,reject.
      \item Go back to the beginning and search for another 1, if one is found, write x on the cell; if none is found, reject.
      \item Go back to step 1
      \item If there are only x’s then accept; otherwise, reject
    \end{enumerate}

  \item (10 points) Problem 3.15 (b), (c) and (e) on page 161.
    \begin{itemize}
     \item Concatenation: A TM $M$ can be constructed to decide the concatenation of two decidable languages $AB$. For the input string
     $w$, we can partition $w$ into $w_a$ and $w_b$ for every possible way to cut $w$ into two parts. TM $M_A$ will take $w_a$ and TM $M_B$ will take $w_b$. If both
     $M_A$ and $M_B$ accept, then M will accept, otherwise it will reject.
     
     \item Star: A TM $M$ can be constructed to decide $L*$. If the input string $w$ is empty, $M$ will accept. Then for each way to divide
     $w$, the substrings will be input to TM $M_L$. If $M$ accepts all substrings in any one partition, then $M$ will accept $L^*$,
     otherwise it will reject.
     
     \item Intersection: A TM $M$ can be constructed to decide $A \cap B$. TM $M_A$ decides $A$ and TM $M_B$ will decide $B$. We then use
     input string $w$ as input for $M_A$. If it is not rejected, it is input to $M_B$. If it is accepted in $M_B$ it is accepted, otherwise
     it be rejected.
    \end{itemize}

      
  \item (10 points) Problem 3.16 (b), (c) and (d) on page 161.
  \begin{itemize}
    \item Concatenation: A TM $M$ can be constructed to recognize the language $AB$. TM $M_A$ will recognize $A$ and TM $M_B$ will
	  recognize $B$. Input string $w$ can beb non-deterministically partitioned to $w_a$ and $w_b$. $w_a$ will be used for $M_A$ and
	  $w_b$ will be used for $M_B$. While $w_a$ is running, $w_b$ should be run. If $M_B$ accepts, then it will be accepted.
    \item Star: A TM $M$ can be constructed to recognize $L*$. If the input string $w$ is empty, $M$ will accept. $w$ is nondeterministically
	  cut into multiple parts, the substrings should be run as input to TM $M_L$. If $M_L$ accepts all the substrings, then $M$ will accept.
    \item Intersection: A TM $M$ can be constructed to recognize $A \cap B$. TM $M_A$ recognizes $A$ and TM $M_B$ recognizes $B$. An input
	  string, $w$, is then run on $M_A$. If $M_A$ accepts $w$, then $w$ will be used as input for $M_B$. If $M_B$ accepts, then M will
	  accept.
  \end{itemize}

\end{enumerate}

\end{document}
