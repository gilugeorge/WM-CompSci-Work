\documentclass[11pt]{article}

\usepackage{exscale}
\usepackage{graphicx}
\usepackage{amsmath}
\usepackage{latexsym}
\usepackage{times,mathptm}
\usepackage{epsfig}
\usepackage{setspace}

\textwidth 6.5truein          
\textheight 9.0truein
\oddsidemargin 0.0in
\topmargin -0.6in

\parindent 0pt          
\parskip 5pt
\def\baselinestretch{1.1}

\begin{document}

\begin{LARGE}
\centerline {\bf CSci 423 Homework 9}
\end{LARGE}
\vskip 0.25cm

\centerline{Due: 1:00 pm, Wednesday, 11/14}
\centerline{Eric Shih}

\begin{enumerate}
  \item (7 points) Exercise 4.3 on page 183. \\
    A decidable language is closed under the complement. TM $M_{\bar{L}}$ which decides $\bar{L}$ will accept when TM $M_{\bar{L}}$ that
    decides $L$ rejects, and reject when $M_{\bar{L}}$ accepts. It has been shown that $E_{DFA}$ is decidable where $E_{DFA}$ = $\{<A> | $
    is a DFA and $L(A)= \emptyset \}$. \\
    On input $<A>$ where $A$ is a DFA:
    \begin{enumerate}
      \item Construct a DFA B that recognizes $\bar{L(A)}$, by swapping accept and non-accepting states.
      \item Run the TM E on input $< B >$, where E is the decider of $E_DFA$.
      \item If E accepts, accept. If E rejects, reject.
    \end{enumerate}

  \item (7 points) Exercise 4.6 on page 183. \\
    Proof by contradiction. Assume that $B$ is countable. Infinate sequences over ${0,1}$ can be represented by $N = {1,2,3,...}$. A
    $f(n)$ can be mapped where $n \in N$. $f(n) = (w_{n1},w_{n2},w_{n3},...)$ where $w_{ni}$ is the $i$th bit in the $n$th sequence. Using
    diagonalization, it is possible to find a sequence over ${0,1}$ that is not in $f(n)$, where the $i$th bit is the opposite of the $i$th
    sequence. This means that the sequences are $\notin$ $B$, giving us a contradiction, proving $B$ to be uncountable. \\
    $
    \begin{array}{l|l}
     $n$ & f(n) \\ \hline
     $1$ & (w_{11}, w_{12},w_{13},w_{14},...) \\
     $2$ & (w_{21}, w_{22},w_{23},w_{24},...) \\
     $3$ & (w_{31}, w_{32},w_{33},w_{34},...) \\
     \vdots & \vdots
    \end{array}
    $
  \item (7 points) Exercise 4.7 on page 183. \\
  *reference staffwww.dcs.shef.ac.uk/people/J.Marshall/alc/studyguides/Selected.pdf* \\
  Construct a function $f::$ T $\to$ N which is uniquely invertible. Remember the
unique factorisation theorem, which states that every number in N is expressible as a unique
product of primes. Consequently we may choose three arbitrary primes $p \neq q \neq r$ and define our
function $f(i, j , k) = p^iq^jr^k$. From the unique prime factorisation theorem we see that any choice
of $i, j , k$ uniquely determines a value of $f$, and so $f$ is invertible.

  \item (7 points) Problem 4.19 on page 184. \\
  *reference http://www.scribd.com/doc/47299154/Solution−Manual−Introduction−Sipser.pdf*
  For any language A, let $A^R = \{w^R |w \in A\}$. If $< M >$ in $S$, then $L(M) = L(M)^R$ . The following TM T
decides S. T = On the input $< M >$, where M is a DFA:
\begin{enumerate}
 \item Construct DFA N that recognizes $L(M)^R$.
  \item Run TM F from Theorem 4.5 on $< M, N >$, were F is the Turing machine deciding $EQ_{DFA}$.
  \item If F accepts, accept. If F rejects, reject.
\end{enumerate}
  
  \item (12 points) In class, we learned that $A_D$ is non-TR, $A_{TM}$ and $HALT_{TM}$ are TR but non-TD. What can you say 
	about their complements? Are they non-TR, TR but non-TD, or TD? Justify your answers. \\
	
	Both $\bar{A_{TM}}$ and $\bar{HALT_{TM}}$ are not TR since $A_{TM}$ and $HALT_{TM}$ are TR. If both sets were TR, then
	they would also be TD. \\
	$\bar{A_D}$ is not TD, according to the closure property, because $A_D$ is not TD. $\bar{A_D}$ is, on the other hand, TR.
	In the case $A_D = \{w_i | w_i \notin L(M_i) \}$, its complement $\bar{A_D} = \{w_i | w_i \in L(M_i) \}$. A TM R can be
	constructed to take $w_i$ as input, then use the string to code TM $M_i$. Both are then fed into $A_TM$, which will accept
	if it accepts. This shows that $\bar{A_D}$ is TR.
\end{enumerate}

\pagebreak
\setlength{\parindent}{1cm}
\centerline{\bf Reading Summary 4}

\begin{spacing}{1.5}
 \indent The turing machine is far from being bettered by other models. Over 75 years, the Turing Machine is still considered the best machine
 model available. It can be used to model both modern computing and the natural world. What we have learned over the years in computing is a
 result of the model. \\
 \indent The Turing machine was so special because it disembodied the machine and broadened the view of computation. It brought the view away
 from what goes specifically into the machine, but how the machine reacts and works as a whole. There are also challenges to the standard model,
 which all have an impact on universality; they are labelled into sections called reductionalists, impressionists, remodelers, and incomputability
 theorists(recursion theorists). The most generally respected challenge to the model, reductionalists, is explained by saying ``one can apply 
 it ever more widely to what we observe in nature, and let it mold our expectations of what computing machines can achieve. \\
 \indent The impressionists are a large force in the community of real-world ''computation``. Real-world computation is how nature and people in
 the real world compute. The old definition of computation being steps carried out by a computer is now being challenged by the idea that computation
 is something can be a non-stop natural process. Some believe that this old idea is a result of a ''Mathematical Bias`` because the founders of
 computing were mathematicians. \\
 \indent To capture much of the computational aspects of nature, people have turned to creating models to be analyzed. Hypercomputational 
 packages is one model that is subject to much disagreement. Some think that this idea of computational models that go beyond the Turing Machine
 are impossible and wrong, while others believe that hypercomputational packages are the way forward. \\
 \indent Finally, the halting set is used by many scholars to explain incomputability. They believe that incomputability is simply the halting set
 with an existential quantifier added. As it is described, ''All the fuss and descriptive variety associated with incomputability in nature and 
 its models merely cloak avatars of halting sets for different machines.``  \\
 \indent Even with the amount of discourse between scholars regarding the future of the Turing Machine, there is no doubt that it's impact is huge among
 the subject of computability and will continue to play a factor in the future.
\end{spacing}


\end{document}
