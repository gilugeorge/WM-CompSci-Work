\documentclass[11pt]{article}

\usepackage{exscale}
\usepackage{graphicx}
\usepackage{amsmath}
\usepackage{latexsym}
\usepackage{times,mathptm}
\usepackage{epsfig}
\usepackage{setspace}

\textwidth 6.5truein
\textheight 9.0truein
\oddsidemargin 0.0in
\topmargin -0.6in

\parindent 0pt
\parskip 5pt
\def\baselinestretch{1.1}

\begin{document}

\begin{LARGE}
\centerline {\bf CSci 427 Homework 1}
\end{LARGE}
\vskip 0.25cm

\centerline{Due: 5:00 pm, Monday, 1/21}
\centerline{Eric Shih}

\setlength{\parindent}{1cm}
\indent This assignment took around 2 hours total to finish. No additional files or methods were implemented. All features provided seemed to work
correctly. \\
\indent This assignment was a fairly straightforward introduction to how future assignments will look and work. The comments and lecture 1
were sufficient enough resources to implement all required functions except for normalize(). For normalize(), a simple wolfram alpha search
was all that was needed to implement the method. Finally, Any results that did not have a provided answer was checked using wolfram alpha
to run the respective operation.
\\ \\
\textit{Output:} \\
eric@ubuntu:~/Documents/WM-CompSci-Work/cs427/HW1$\$$ ./hw1 \\
Vector u = [1,2,3] \\
Vector v = [1,0,0] \\
Vector w = [0,1,0] \\
u + v    = [2,2,3] should be: [2,2,3] \\
u - w    = [1,1,3] should be: [1,1,3] \\
u * w    = [0,2,0] should be: [0,2,0] \\
u * 0.5f = [0.5,1,1.5] should be: [0.5,1,1.5] \\
2.0f * ((v+w) / u)) = [2,1,0] \\
v += w,   v = [1,1,0] should be: [1,1,0] \\
v *= u,   v = [1,2,0] should be: [1,2,0] \\
v /= u,   v = [1,1,0] should be: [1,1,0] \\
v -= w,   v = [1,0,0] should be: [1,0,0] \\
u.squared$\_$length() = 14 should be: 14 \\
v.dot(u) = 1 \\
v.dot(w) = 0 \\
w.dot(v) = 0 \\
v.normalize() = [1,0,0] should be: [1,0,0] \\
u.normalize() = [0.267261,0.534522,0.801784] \\
v.cross(w) = [0,0,1] should be: [0,0,1] \\
u.cross(w) = [-3,0,1] \\
u.cross(u) = [0,0,0] \\

\end{document}
