\documentclass[11pt]{article}

\usepackage{exscale}
\usepackage{graphicx}
\usepackage{amsmath}
\usepackage{latexsym}
\usepackage{times,mathptm}
\usepackage{epsfig}
\usepackage{setspace}

\textwidth 6.5truein
\textheight 9.0truein
\oddsidemargin 0.0in
\topmargin -0.6in

\parindent 0pt
\parskip 5pt
\def\baselinestretch{1.1}

\begin{document}

\begin{LARGE}
\centerline {\bf CSci 427 Homework 5}
\end{LARGE}
\vskip 0.25cm

\centerline{Due: 12:00 pm, Tuesday, 2/26}
\centerline{Eric Shih}

\setlength{\parindent}{1cm}
\indent This assignment took approximately 7 hours to complete. No additional files or methods were implemented. Image 1 and 3 are off
by 5000 pixels and 1000 pixels respectively. I think this is because the preciseness of the coordinates that I provide $\_$texture. \\
\indent The role of the transformation in the constructor of the texture projections is to transform the newly constructed triangles in the
projection. This is useful when different parts of the scene get transformed separately. The texturedAlbedo will load in the image when
there is one available. This image is then evaluated as a material, where the colors are returned and the coordinates matched. This gets
repeated with the other types of materials depending on the type of projection being done.
\end{document}