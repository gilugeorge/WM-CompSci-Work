\documentclass[11pt]{article}

\usepackage{exscale}
\usepackage{graphicx}
\usepackage{amsmath}
\usepackage{latexsym}
\usepackage{times,mathptm}
\usepackage{epsfig}
\usepackage{setspace}

\textwidth 6.5truein
\textheight 9.0truein
\oddsidemargin 0.0in
\topmargin -0.6in

\parindent 0pt
\parskip 5pt
\def\baselinestretch{1.1}

\begin{document}

\begin{LARGE}
\centerline {\bf CSci 427 Homework 2}
\end{LARGE}
\vskip 0.25cm

\centerline{Due: 12:00 pm, Monday, 1/29}
\centerline{Eric Shih}

\setlength{\parindent}{1cm}
\begin{spacing}{1.5}
\indent This assignment took approximately 5 hours to finish. No additional files or methods were implemented, nor were any features
incorrect. \\
\indent Overall, the assignment took a bit of time to really understand the math behind the triangle intersection, but once the pieces were sorted
out, the implementation was fairly straightforward. A reference paper from Curless was useful in determining whether or not a point was
within the triangle; otherwise, the slides and book were sufficient enough to easily implement the images. \\
\indent generateViewRay() is the method that creates the vector properties for the specific point in question. The coordinates to the point
are sent in as arguments.  First the x vector is calculated using arguments provided. Then the aspect ratio is calculated using the field of
view arguments provided. Both of these will be used in conjunction with other provided arguments to calculate the U and V vectors that are
necessary to compute the viewplane. U represents the horizontal view, while V is the vertical view. The point on the viewplane is used so
that the resulting vector will point in the correct direction towards the plane. Once the direction is calculated, the vector is done, with
the origin having already been provided.
\end{spacing}
\end{document}